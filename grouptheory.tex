\section{Group Theory}
\subsection{The Commutator Subgroup}
Let $G$ be a group, $S \subseteq G$ a set of elements in $G$. Then $\langle S \rangle$\index{$\langle S \rangle$} or $\gp\{S\}$\index{$\gp\{S\}$} denotes the subgroup generated by $S$ in $G$\index{subgroup generated by $S$ in $G$}:
\begin{align*}
	\langle S \rangle &= \{s_1^{\varepsilon_1} \dots s_k^{\varepsilon_k} \mid s_i \in S,\ \varepsilon_i \in \{\pm 1\},\ k \geq 0\} \\
		& = \bigcap_{H \leq G : S \subseteq H}{H}.
\end{align*}
Note: If $S = \{s\}$, we get the cyclic subgroup generated by $s$.

If $\langle S \rangle = G$, we say that $S$ is a \defn{generating set} for $G$.

\begin{examples}\hfill
	\begin{enumerate}
		\item The set of all cycles is a generating set for the symmetric group $S_n$.
		\item The set of all transpositions is a generating set for $S_n$.
		\item The transpositions $(12), (23), (34), \dots, (n - 1\ n)$ form a generating set for $S_n$.
		\item The transposition $(12)$ together with the full cycle $(12 \dots n)$ forms a generating set for $S_n$.
		\item The set of all $3$-cycles in $S_n$ is a generating set for the alternating group $A_n$.
	\end{enumerate}
	Examples 1, 2, 3 and 5 were covered in the Algebraic Structures and Group Theory course units. Example 4 follows from 3 upon noting that
	\[
		(12 \dots n)^k (12) (12 \dots n)^{-k} = (k + 1\ k + 2),
	\]
	for $k = 1, 2, \dots, n - 2$.
\end{examples}

\begin{definition}
	Let $x, y \in G$. The element
	\[
		[x, y] = x^{-1}y^{-1}xy
	\]\index{$[x, y]$}
	is called the \defn{commutator} of $x$ and $y$ in $G$.
\end{definition}

\subsubsection{Properties of the commutator}
\begin{itemize}
	\item $[x, y] = 1$ if and only if $xy = yx$;
	\item $G$ is abelian if and only if $[x, y] = 1$ for all $x, y \in G$;
	\item $[x, y]^{-1} = [y, x]$.
\end{itemize}

\begin{definition}
	The subgroup of $G$ that is generated by all commutators in $G$ is called the \defn{commutator subgroup} or \defn{derived subgroup}. We denote it by $G' = \langle [x, y] \mid x, y \in G \rangle$.
\end{definition}

\begin{lemma}\label{lem:commutator-results}
	Let $G$ be an arbitrary group. Then
	\begin{enumerate}
		\item $G$ is abelian if and only if $G' = \{1\}$;
		\item $G'$ is a normal subgroup of $G$, i.e. $G' \unlhd G$;
		\item If $\varphi : G \to H$ is an epimorphism\footnote{An epimorphism is a surjective homomorphism.} of groups, then $\varphi(G') = H'$;
		\item If $N \unlhd G$ then $G / N$ is abelian if and only if $G' \subseteq N$. In particular, $G / G'$ is abelian.
	\end{enumerate}
	\begin{proof}\hfill
		\begin{enumerate}
			\item $G$ is abelian $\iff xy = yx$ for all $x, y \in G \iff [x, y] = 1$ for all $x, y \in G \iff G' = \{1\}$.
			\item We aim to show that $G'$ is closed under conjugation.
			
			First note that the inverse of a commutator is a commutator. Now consider
			\begin{align*}
				g^{-1}[x, y]g &= g^{-1}x^{-1}y^{-1}xyg \\
					&= g^{-1}x^{-1}g g^{-1}y^{-1}g g^{-1}xg g^{-1}yg \\
					&= [g^{-1}xg, g^{-1}yg],
			\end{align*}
			so we see that the conjugate of a commutator is also a commutator. But this means that conjugates of products of commutators are products of commutators, since
			\[
				g^{-1} c_1 c_2 \dots c_k g = g^{-1} c_1 g g^{-1} c_2 g g^{-1}\dots g g^{-1}c_k g.
			\]
			Hence $G'$ is closed under conjugation, so $G' \unlhd G$.
			\item Suppose $\varphi : G \to H$ is an epimorphism of groups. Since $\varphi([x, y]) = [\varphi(x), \varphi(y)]$, we have $\varphi(G') \subseteq H'$. Since $\varphi$ is surjective, each commutator in $H$ is the image of a commutator in $G$. Hence $\varphi(G') = H'$.
			\item Let $\nu : G \to G / N$ be the natural homomorphism.\footnote{The natural homomorphism is defined by $\nu : G \to G / N : g \mapsto gN$. Note that $\ker{\nu} = N$.} By (i), the factor group $G / N$ is abelian if and only if $(G / N)' = \{1\}$. By (iii), we have $(G / N)' = \nu(G')$, so $G / N$ is abelian $\iff \nu(G') = \{1\} \iff G' \subseteq \ker{\nu} = N$. Hence $G / N$ is abelian if and only if $G' \subseteq N$.
		\end{enumerate}
	\end{proof}
\end{lemma}

\begin{examples}
	\begin{enumerate}
		\item $S'_n = A_n$. This is because, in $S_n$ we have
		\[
			[(ab), (ac)] = (ab)(ac)(ab)(ac) = (abc).
		\]
		Hence $S'_n$ contains all 3-cycles, which generate $A_n$. Hence $A_n \subseteq S'_n$. But the quotient $S_n / A_n$ is abelian (in fact, cyclic of order 2). By Lemma \ref{lem:commutator-results} (iv), $S'_n \subseteq A_n$ and hence $S'_n = A_n$.
		\item For $n \geq 5$, $A'_n = A_n$. This is because, for $n \geq 5$, $A_n$ is a non-abelian simple group. The only normal subgroups of the simple group $A_n$ are $\{1\}$ and $A_n$. Since $A_n$ is non-abelian, $A'_n \neq \{1\}$ (by Lemma \ref{lem:commutator-results} (i)) and hence $A'_n = A_n$.
		\item $A'_4 = \{1, (12)(34), (13)(24), (14)(23)\}$. Indeed, we have
		\begin{align*}
			[(abc), (abd)] &= (acb)(adb)(abc)(abd) \\
				&= (ab)(cd).
		\end{align*}
		Hence $A'_n$ contains all double transpositions. If $n = 4$ these form, together with the identity, a normal subgroup of order 4. The quotient of $A_4$ by this normal subgroup has order 3, so it is abelian. So $A'_4$ coincides with this normal subgroup.
	\end{enumerate}
\end{examples}

\begin{definition}
	The higher commutator subgroups $G^{(n)}$ of a group $G$ are defined inductively by setting $G^{(n)} = (G^{(n - 1)})'$ starting with $G^{(1)} = G'$. The first three are usually written $G'$, $G''$ and $G'''$. The \defn{derived series} of a group $G$ is the series
\[
	\dots \unlhd G^{(n)} \unlhd G^{(n - 1)} \unlhd \dots \unlhd G''' \unlhd G'' \unlhd G' \unlhd G.
\]
\end{definition}

\subsection{Solvable Groups}
\begin{definition}
	A group $G$ is \emph{solvable}\index{solvable!group} (or \emph{soluble}\index{soluble group}) if there is a finite chain of subgroups
	\begin{equation}\label{eq:chain-10}
		\{1\} = H_r \leq H_{r - 1} \leq \dots \leq H_2 \leq H_1 \leq H_0 = G
	\end{equation}
	such that for $i = 1, 2, \dots, r$,
	\begin{enumerate}
		\item each $H_i$ is \emph{normal} in $H_{i - 1}$, and
		\item the factor group $H_{i - 1} / H_i$ is \emph{abelian}.
	\end{enumerate}
\end{definition}

\begin{lemma}
	A group $G$ is solvable if and only if $G^{(k)} = \{1\}$ for some $k \geq 1$.
	\begin{proof}
		Suppose $G^{(k)} = \{1\}$. Then the derived series is a finite chain with the required properties.
		
		Now suppose $G$ is solvable with a chain (\ref{eq:chain-10}). We proceed by induction on $r$.
		
		If $r = 1$ then $H_1 = \{1\}$ and $G = H_0 = H_0 / H_1$ is abelian, so $G' = \{1\}$ (by Lemma \ref{lem:commutator-results} (i)).
		
		If $r > 1$ then we apply the induction hypothesis to $H_1$, so $H_1^{(k)} = \{1\}$ for some $k$. Since $G / H_1$ is abelian, we have $G' \subseteq H_1$. Now, it is clear that if $A \leq B$ then $A' \leq B'$. But then $G'' \subseteq (H_1)'$, $G''' \subseteq (H_1)''$ and so on, so that $G^{(r + 1)} \subseteq H_r = \{1\}$. Hence $G^{(r + 1)} = \{1\}$ as required.
	\end{proof}
\end{lemma}
