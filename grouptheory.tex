\section{Group Theory}
\subsection{The Commutator Subgroup}
Let $G$ be a group, $S \subseteq G$ a set of elements in $G$. Then $\langle S \rangle$\index{$\langle S \rangle$} or $\gp\{S\}$\index{$\gp\{S\}$} denotes the subgroup generated by $S$ in $G$\index{subgroup generated by $S$ in $G$}:
\begin{align*}
	\langle S \rangle &= \{s_1^{\varepsilon_1} \dots s_k^{\varepsilon_k} \mid s_i \in S,\ \varepsilon_i \in \{\pm 1\},\ k \geq 0\} \\
		& = \bigcap_{H \leq G : S \subseteq H}{H}.
\end{align*}
Note: If $S = \{s\}$, we get the cyclic subgroup generated by $s$.

If $\langle S \rangle = G$, we say that $S$ is a \defn{generating set} for $G$.

\begin{examples}\hfill
	\begin{enumerate}
		\item The set of all cycles is a generating set for the symmetric group $S_n$.
		\item The set of all transpositions is a generating set for $S_n$.
		\item The transpositions $(12), (23), (34), \dots, (n - 1\ n)$ form a generating set for $S_n$.
		\item The transposition $(12)$ together with the full cycle $(12 \dots n)$ forms a generating set for $S_n$.
		\item The set of all $3$-cycles in $S_n$ is a generating set for the alternating group $A_n$.
	\end{enumerate}
	Examples 1, 2, 3 and 5 were covered in the Algebraic Structures and Group Theory course units. Example 4 follows from 3 upon noting that
	\[
		(12 \dots n)^k (12) (12 \dots n)^{-k} = (k + 1\ k + 2),
	\]
	for $k = 1, 2, \dots, n - 2$.
\end{examples}
