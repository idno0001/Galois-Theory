\section{Preliminaries}
\subsection{Fields}
\begin{definition}
	A \emph{field} is a set $F$ with a pair of binary operations $+$ and $\times$, called \emph{addition} and \emph{multiplication}, such that
	\begin{enumerate}
		\item $F$ is an abelian group with respect to $+$. That is, $+$ is associative and commutative, there exists a neutral element $0$ such that $a + 0 = a$ for all $a \in F$, and for all $a \in F$ there exists an element $(-a) \in F$ such that $a + (-a) = 0$;
		\item The multiplication on $F$ is associative and commutative, there exists a multiplicative identity element $1 \neq 0$ such that $1 \cdot a = a$ for all $a \in F$, and for all $a \in F$, $a \neq 0$, there exists an element $a^{-1} \in F$ such that $a \cdot a^{-1}$;
		\item For all $a, b, c \in F$, $a(b + c) = ab + ac$ (distributivity).
	\end{enumerate}
\end{definition}

\begin{example}
	$\Q$, $\R$, $\C$ and $\Z_p = \Z / p\Z$, where $p$ is a prime, are all fields.
\end{example}

\begin{remarks} \hfill
	\begin{itemize}
		\item $0$ and $1$ are unique, and for all $a \in F$, $-a$ is unique. If $a \neq 0$ then $a^{-1}$ is unique.
		\item The condition $1 \neq 0$ is equivalent to the condition that $F$ has at least two elements.
	\end{itemize}
\end{remarks}

\begin{definition}
	Suppose $F$ is a field and $E \subset F$. If $E$ is a field with respect to the binary operations of $F$, then we say that $E$ is a \emph{subfield}.
	
	In fact, a subset $E$ of a field $F$ is a subfield if and only if
	\begin{itemize}
		\item $0, 1 \in E$;
		\item $E$ is closed under $+$ and $\times$; and
		\item $E$ is closed under taking additive and multiplicative inverses.
	\end{itemize}
\end{definition}
