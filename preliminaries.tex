\section{Preliminaries}
\subsection{Fields}
\begin{definition}
	A \emph{field} is a set $F$ with a pair of binary operations $+$ and $\times$, called \emph{addition} and \emph{multiplication}, such that
	\begin{enumerate}
		\item $F$ is an abelian group with respect to $+$. That is, $+$ is associative and commutative, there exists a neutral element $0$ such that $a + 0 = a$ for all $a \in F$, and for all $a \in F$ there exists an element $(-a) \in F$ such that $a + (-a) = 0$;
		\item The multiplication on $F$ is associative and commutative, there exists a multiplicative identity element $1 \neq 0$ such that $1 \cdot a = a$ for all $a \in F$, and for all $a \in F$, $a \neq 0$, there exists an element $a^{-1} \in F$ such that $a \cdot a^{-1}$;
		\item For all $a, b, c \in F$, $a(b + c) = ab + ac$ (distributivity).
	\end{enumerate}
\end{definition}

\begin{example}
	$\Q$, $\R$, $\C$ and $\Z_p = \Z / p\Z$, where $p$ is a prime, are all fields.
\end{example}

\begin{remarks} \hfill
	\begin{itemize}
		\item $0$ and $1$ are unique, and for all $a \in F$, $-a$ is unique. If $a \neq 0$ then $a^{-1}$ is unique.
		\item The condition $1 \neq 0$ is equivalent to the condition that $F$ has at least two elements.
	\end{itemize}
\end{remarks}

\begin{definition}
	Suppose $F$ is a field and $E \subset F$. If $E$ is a field with respect to the binary operations of $F$, then we say that $E$ is a \emph{subfield}.
	
	In fact, a subset $E$ of a field $F$ is a subfield if and only if
	\begin{itemize}
		\item $0, 1 \in E$;
		\item $E$ is closed under $+$ and $\times$; and
		\item $E$ is closed under taking additive and multiplicative inverses.
	\end{itemize}
\end{definition}

\begin{definition}
	The \emph{characteristic} of a field $F$ is the  smallest $n \in \N$ such that $n \cdot 1 = \underbrace{1 + \ldots + 1}_{n \text{ times}} = 0$.
	
	If there is no such $n$, we say that $F$ has characteristic $0$.
\end{definition}

\begin{proposition}
	If $F$ has positive characteristic, then the characteristic is a prime number.
	\begin{proof}
		Suppose we have $n \cdot 1 = 0$ with $n = m \cdot k$ for some $m, k \in \N$. Then
		\[
			n \cdot 1 = \underbrace{1 + \ldots + 1}_{n \text{ times}} = (\underbrace{1 + \ldots + 1}_{m \text{ times}})(\underbrace{1 + \ldots + 1}_{k \text{ times}}) = (m \cdot 1)(k \cdot 1) = 0.
		\]
	\end{proof}
	Now if the product of 2 elements is $0$ in a field, then at least one of these elements must be zero (since a field has no zero divisors). So if we take the prime factorisation of $n$, the result follows.
\end{proposition}

\begin{remarks}\hfill
	\begin{itemize}
		\item If a field $F$ has characteristic $p$, $p$ a prime, then $p \cdot a = 0$ for all $a \in F$:
		\[
			p \cdot a = \underbrace{a + \ldots + a}_{p \text{ times}} = a(\underbrace{1 + \ldots + 1}_{p \text{ times}}) = a \cdot 0 = 0.
		\]
		\item $\Q$, $\R$ and $\C$ have characteristic $0$.
		\item $\Z_p$ has characteristic $p$.
	\end{itemize}
\end{remarks}

\begin{definition}
	The smallest subfield of a field $F$ is called the \emph{prime subfield}.
\end{definition}

\begin{proposition}
	The prime subfield of any field is isomorphic to either $\Q$ or $\Z_p$ for some prime $p$.
	\begin{proof}
		A subfield contains $0$ and $1$, hence it contains all elements $n \cdot 1 = \underbrace{1 + \ldots + 1}_{n \text{ times}}$. It is easy to see the map $\varphi : \Z \to F$ defined by $n \mapsto n \cdot 1$ extends to a (ring) homomorphism (see the First Isomorphism Theorem for rings).
		
		If $\text{char}(F) = 0$, then $\text{ker}(\varphi) = \{0\}$. Therefore $F$ contains an isomorphic copy of $\Z$, but then it also contains an isomorphic copy of $\Q$ (since $\Q$ contains all the multiplicative inverses of $\Z$). Again, see the First Isomorphism Theorem for rings.
		
		If $\text{char}(F) = p$ then $\text{ker}(\varphi) = p\Z$. Therefore $F$ contains an isomorphic copy of $\Z_p = \Z / p\Z$, which is  a field. So this must be the prime subfield.
	\end{proof}
\end{proposition}

\begin{notation}
	In field theory, if $a neq 0$, we often write $a^{-1} = \frac{1}{a}$ and $\frac{b}{a} = a^{-1}b$.
\end{notation}

\subsection{Homogeneous Linear Systems}
Suppose we have a field $F$. Then the equation $ax = b$, $a, b \in F$, $a \neq 0$, has a unique solution in F. This is given by $x = a^{-1}b$. This is one of the most important features of a field $F$.

In particular, $x = 0$ is the only solution of $ax = 0$, ($a \neq 0$).

More generally, the system
\begin{equation} \label{eq:hom-system}
	\left.
	\begin{matrix}
		a_{1 1} x_1 &+& a_{1 2} x_2 &+& \ldots &+& a_{1 n} x_n &=& 0 \\
		\vdots		& & \vdots	    & & \vdots & & \vdots	   & & \vdots \\
		a_{m 1} x_1 &+& a_{m 2} x_2 &+& \ldots &+& a_{m n} x_n &=& 0
	\end{matrix}
	\right\}
	\qquad \text{Maxwell's equations}
\end{equation}
($a_{ij} \in F$) always has the \emph{trivial} solution $x_1 = x_2 = \ldots = x_n = 0$. Any other solution is called \emph{non-trivial}.

\begin{theorem}
	A homogeneous system (\ref{eq:hom-system}) has a non-trivial solution if $n > m$. (i.e. The number of variables is greater than the number of equations.)
	\begin{proof}
		
	\end{proof}
\end{theorem}
